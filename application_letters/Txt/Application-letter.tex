\section{Application for HPI Professorship in Software Engineering \& AI}

I am writing to apply for the W3 Professorship position in Software Engineering and AI at the Hasso Plattner Institute.
As an Associate Professor at the University of Bari with expertise in AI and software engineering, I am excited about the opportunity to contribute to HPI's mission of advancing AI-driven software development approaches.

\textbf{Research Profile and Contributions}.
My research program focuses on the intersection of AI and software engineering, precisely matching HPI's requirements for this position. 
I have developed several significant AI models and tools for software engineering, including EMTk (Emotion-Mining Toolkit) for analyzing developer communications and Pynblint for quality assurance in Python notebooks (SoftwareX, 2024).
My recent work investigates the self-admitted use of generative AI in open-source projects (IEEE TSE, under review, 2025), contributing to our understanding of how AI is transforming software development practices. 
As co-author of \textit{The Copenhagen Manifesto} on human-centered generative AI in software engineering (JSS, 2024), I have helped shape the discourse on responsible AI integration in software development. 
My research has generated significant impact with an \textit{h}-index of 30 and over 3,100 citations (Google Scholar), with publications appearing regularly in top-tier venues including IEEE Transactions on Software Engineering, ACM Transaction on Software Engineering and Methodology, Empirical Software Engineering, Journal of Systems and Software, and Information and Software Technology.
As co-founder and CEO of PeoplewareAI, a university spin-off company, I have demonstrated success in transferring research results in AI and software engineering to industry applications. 

\textbf{Teaching and Mentorship}.
Throughout my academic career, I have demonstrated excellence in both undergraduate and graduate education. 
My teaching portfolio spans from foundational Computer Networks courses to specialized PhD-level seminars, consistently receiving teaching evaluations above 95\% satisfaction. 
I have successfully supervised several PhD students working on cutting-edge research in AutoML, AI-enabled systems, and OSS development. 
My approach to curriculum development emphasizes practical application alongside theoretical foundations, as evidenced by my innovative courses in \textit{Social Computing} and \textit{Software Solutions for Reproducible Experiments}.
These courses combine traditional lectures with hands-on project work, preparing students for both academic research and industry challenges.

\textbf{Research Collaborations}.
My research has been enriched through sustained international collaborations in software engineering and AI.
At Carnegie Mellon University's STRUDEL lab, I investigated developer personalities in large-scale software ecosystems (IST, 2019). 
Through collaboration with Northern Arizona University's RESHAPE group, I studied developer disengagement in open source projects (EMSE, 2022), work that led to a joint NSF-MUR project proposal on AI applications.
My partnerships with University of Victoria's CHISEL group and University of Oulu's M-Group have advanced research in human aspects of AI-enabled systems and MLOps practices (ICSA, 2024), respectively.

\textbf{Research and Industry Funding}.
My funding experience spans both academic research and industry innovation. As an academic researcher, I am currently leading the University of Bari's participation in \textit{DisTrac} (€215K MUR funding proposal), a joint NSF-MUR project focusing on analyzing developer disengagement in open source projects.\\
At the national level, I serve as workpackage leader in two major research initiatives in healthcare AI. In \textit{DARE} (€130.5M total funding), a national project creating a distributed knowledge community for digital preventive healthcare, my responsibilities include designing MLOps practices and developing technical solutions for secure deployment of AI models in healthcare settings. In \textit{SERICS-SuReCare} (€114.5M total funding), focused on cybersecurity challenges in remote healthcare systems, I coordinate the Detection-Response-Prevention workpackage, developing approaches to control ML models transitioning into production.\\
My past experience in managing international research funding includes serving as scientific coordinator for the Italian unit in PRONEM (Brazilian Ministry of Education), where I led research on natural language processing for global software development. I have also contributed to European projects such as \textit{INTERSOCIAL} (EU INTERREG Greece-Italy) and \textit{OpEn} (Erasmus+ Program), where I successfully led work packages focused on developing innovative software engineering solutions.\\
As co-founder and CEO of PeoplewareAI, a university spin-off company, I have secured private investment for translating academic research into industry applications. The company is currently participating in \textit{ARIANNA} (€1M regional funding proposal), contributing expertise in AI applications for virtual meetings platforms. The company has successfully developed commercial AI solutions for behavioral analysis and MLOps, with a growing focus on healthcare applications through national funding partnerships.

\textbf{Proposed Research Program}. TODO.
Looking ahead, I am  excited about advancing the integration of AI technologies into software engineering practices while ensuring a human-centered approach. At HPI, I envision establishing a research program focusing on.

\begin{itemize}
\item Developing robust frameworks for evaluating and implementing AI-powered software engineering tools
\item Investigating the impact of generative AI on software development practices
\item Creating novel approaches for AI-assisted code quality improvement and developer support
\end{itemize}

This program would build on my existing and ongoing research while opening new avenues specifically enabled by HPI's resources.

\textbf{Institutional Fit}. 
The unique position of HPI at the intersection of academic research and industry collaboration provides an ideal environment for my research agenda. 
The emphasis on high standards in both research and teaching at the institute aligns with my academic approach.
The strong connections to industry at HPI align also with my experience in translating research into practical applications. 
Ovefrall, this position at HPI represents an exciting opportunity to shape and advance the field of AI-driven software engineering while also contributing to the tradition of excellence at the institute.

I look forward to discussing how my expertise and vision align with HPI's goals in more detail.