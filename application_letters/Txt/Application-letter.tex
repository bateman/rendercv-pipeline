\section{Application for HPI Professorship in Software Engineering \& AI}

I am writing to apply for the W3 Professorship position in Software Engineering and AI at the Hasso Plattner Institute.
As an Associate Professor at the University of Bari with expertise in AI and software engineering, I am excited about 
the opportunity to contribute to HPI's mission of advancing AI-driven software development approaches.

\textbf{Research Profile and Contributions}:
My research program focuses on the intersection of AI and software engineering, precisely matching HPI's requirements for this position. 
I have developed several significant AI models and tools for software engineering, including EMTk (Emotion-Mining Toolkit) for analyzing developer communications and Pynblint for quality assurance in Python notebooks (SoftwareX, 2024).
My recent work investigates the self-admitted use of generative AI in open-source projects (IEEE TSE, under review, 2025), contributing to our understanding of how AI is transforming software development practices. 
As co-author of ``The Copenhagen Manifesto'' on human-centered generative AI in software engineering (JSS, 2024), I have helped shape the discourse on responsible AI integration in software development. 
My research has generated significant impact with an h-index of 30 and over 3,100 citations (Google Scholar), with publications appearing regularly in top-tier venues including IEEE Transactions on Software Engineering, ACM Transaction on Software Engineering and Methodology, Empirical Software Engineering, Journal of Systems and Software, and Information and Software Technology.
As co-founder and CEO of PeoplewareAI, a university spin-off company, I have demonstrated success in transferring research results in AI and software engineering to industry applications. 

\textbf{Teaching and Mentorship}:
Throughout my academic career, I have demonstrated excellence in both undergraduate and graduate education. 
My teaching portfolio spans from foundational Computer Networks courses to specialized PhD-level seminars, consistently receiving teaching evaluations above 95\% satisfaction. 
I have successfully supervised several PhD students working on cutting-edge research in AutoML, AI-enabled systems, and OSS development. 
My approach to curriculum development emphasizes practical application alongside theoretical foundations, as evidenced by my innovative courses in `Social Computing' and `Software Solutions for Reproducible Experiments.'
These courses combine traditional lectures with hands-on project work, preparing students for both academic research and industry challenges.

\textbf{Research Collaborations}:
My international research network reflects a deep commitment to global collaboration in software engineering research. Extended research visits at Carnegie Mellon University's STRUDEL lab, University of Victoria's CHISEL group, and University of Oulu's M-Group have led to significant outcomes in human-centered software engineering and AI applications. These collaborations have yielded influential publications, including work on developer personalities in large-scale software ecosystems with CMU (IST, 2019) and studies on MLOps practices with Oulu (ICSA, 2024). Through these partnerships, I have developed expertise in managing distributed research teams and fostering cross-cultural academic collaboration - skills that would enhance HPI's international research initiatives.

\textbf{Research Funding}:
My research has attracted substantial external funding through various international projects, including the EU Erasmus+ Program and national research initiatives.

\textbf{Vision | Future Directions}:
Looking ahead, I am particularly excited about advancing the integration of AI technologies into software engineering practices while ensuring a human-centered approach. At HPI, I would aim to establish a research program focusing on:

\begin{itemize}
\item Developing robust frameworks for evaluating and implementing AI-powered software engineering tools
\item Investigating the impact of generative AI on software development practices
\item Creating novel approaches for AI-assisted code quality improvement and developer support
\end{itemize}

I am impressed by HPI's commitment to excellence and its unique position at the intersection of academic research and industry collaboration. The institute's exceptional computing infrastructure and highly selective student body would provide an ideal environment for pursuing these research directions.
Thank you for considering my application. I look forward to discussing how my expertise and vision align with HPI's goals in more detail.