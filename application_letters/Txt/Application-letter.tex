\section{Application for HPI Professorship in Software Engineering \& AI}

I am writing to apply for the W3 Professorship position in Software Engineering and AI at the 
Hasso Plattner Institute. As an Associate Professor at the University of Bari with a strong focus 
on the intersection of AI and software engineering, I am particularly excited about contributing 
to HPI's mission of advancing AI-driven software development approaches.

\textbf{Research Profile and Vision}:
My research program centers on the application of AI techniques to enhance software engineering practices, with particular emphasis on automated machine learning (AutoML), generative AI, and emotion-aware software engineering. This work has resulted in significant contributions to the field, including co-authoring "The Copenhagen Manifesto" on human-centered generative AI in software engineering (JSS, 2024) and leading investigations into industry-leading AutoML tools for software engineering applications (IST, 2024).

My research directly aligns with HPI's focus areas:
\begin{itemize}
\item AI Models and Tools for Software Engineering: I have developed EMTk (Emotion-Mining Toolkit), which provides specialized emotion classifiers for analyzing developers' communication in software projects. This work demonstrates my commitment to creating practical AI tools that enhance software development processes.
\item Generative AI for Software Engineering: My recent work investigates the self-admitted use of generative AI in open-source software projects (TSE, under review, 2025), contributing to our understanding of how AI is transforming software development practices.
\item AI-based Code Analysis and Quality: I've led the development of Pynblint, a quality assurance tool for Python Jupyter notebooks (SoftwareX, 2024), demonstrating my expertise in creating AI-powered tools for code quality improvement.
\end{itemize}

\textbf{Teaching and Mentorship}:
My teaching portfolio includes courses in Social Computing, Computer Networks, and specialized PhD-level courses in Mining Socio-technical Repositories and Software Solutions for Reproducible Experiments. I have successfully supervised multiple PhD students, including recent work on AutoML for Software Engineering and Collaboration Around Computational Notebooks for AI-enabled Systems. My teaching evaluations consistently show high satisfaction rates, often exceeding 95\%. \par

\textbf{Collaborative Vision}:
With extensive experience in international collaboration, including research visits to Carnegie Mellon University, University of Victoria, and University of Oulu, I understand the importance of fostering global research networks. As co-founder and CEO of PeoplewareAI, a university spin-off company, I have demonstrated success in transferring research results in AI and software engineering to industry applications. 

\textbf{Research Funding and Impact}:
My research has attracted substantial external funding through various international projects, including the EU Erasmus+ Program and national research initiatives. My work has generated significant impact with an h-index of 30 and over 3,100 citations (Google Scholar). I have published extensively in top-tier venues including IEEE Transactions on Software Engineering, Journal of Systems and Software, and Information and Software Technology.

\textbf{Future Directions}:
Looking ahead, I am particularly excited about advancing the integration of AI technologies into software engineering practices while ensuring a human-centered approach. At HPI, I would aim to establish a research program focusing on:

\begin{itemize}
\item Developing robust frameworks for evaluating and implementing AI-powered software engineering tools
\item Investigating the impact of generative AI on software development practices
\item Creating novel approaches for AI-assisted code quality improvement and developer support
\end{itemize}

I am impressed by HPI's commitment to excellence and its unique position at the intersection of academic research and industry collaboration. The institute's exceptional computing infrastructure and highly selective student body would provide an ideal environment for pursuing these research directions.
Thank you for considering my application. I look forward to discussing how my expertise and vision align with HPI's goals in more detail