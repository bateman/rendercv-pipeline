\section{Application for HPI Professorship in Software Engineering \& AI}

I am writing to apply for the W3 Professorship position in Software Engineering and AI at the Hasso Plattner Institute.
As an Associate Professor at the University of Bari with expertise in Large Language Models (LLMs) and AI-driven software engineering, I am excited about the opportunity to contribute to HPI's mission of advancing modern software development approaches.

\textbf{Research Profile and Contributions}.
My research program focuses on the intersection of AI and software engineering, with particular emphasis on the transformative impact of Large Language Models.
My recent work investigates the self-admitted use of LLMs and other generative AI tools in open-source projects (IEEE TSE, under review, 2025), providing crucial insights into how these technologies are reshaping development practices. 
As co-author of \textit{The Copenhagen Manifesto} on human-centered generative AI in software engineering (JSS, 2024), I have helped guide the development of frameworks for responsible AI integration in software development. 
I have also developed several significant AI models and tools for software engineering, including \textit{Senti4SD} (EMSE 2018) and \textit{EMTk} (SEmotion@ICSE 2019), for analyzing developer communications, and \textit{Pynblint}, for quality assurance in Python notebooks (SoftwareX, 2024).
My research has generated significant impact with an \textit{h}-index of 30 and over 3,150 citations (Google Scholar), with publications appearing regularly in top-tier venues including IEEE TSE, ACM TOSEM, ESE, JSS, and IST.
As co-founder and CEO of PeoplewareAI, a university spin-off company, I have demonstrated success in transferring research results in AI and software engineering to industry applications. 

\textbf{Teaching and Mentorship}.
Throughout my academic career, I have demonstrated excellence in both undergraduate and graduate education. 
My teaching portfolio spans from foundational Computer Networks courses to specialized PhD-level seminars, consistently receiving teaching evaluations above 95\% satisfaction. 
I have successfully supervised several PhD students working on cutting-edge research in AutoML, AI-enabled systems, and OSS development. 
My approach to curriculum development emphasizes practical application alongside theoretical foundations, as evidenced by my innovative courses in \textit{Social Computing} and \textit{Software Solutions for Reproducible Experiments}.
These courses combine traditional lectures with hands-on project work, preparing students for both academic research and industry challenges.

\textbf{Research Collaborations}.
My research has been enriched through sustained international collaborations in software engineering and AI.
At Carnegie Mellon University's STRUDEL lab, I investigated developer personalities in large-scale software ecosystems (IST, 2019). 
Through collaboration with Northern Arizona University's RESHAPE group, I studied developer disengagement in open source projects (EMSE, 2022), work that led to a joint NSF-MUR project proposal on AI applications.
My partnerships with University of Victoria's CHISEL group and University of Oulu's M-Group have advanced research in human aspects of AI-enabled systems and MLOps practices (ICSA, 2024), respectively.

\textbf{Research and Industry Funding}.
I have secured and managed substantial research funding in software engineering and AI, particularly focusing on the intersection of these fields. Currently, I lead two major workpackages in national healthcare AI initiatives: \textit{DARE} (€130.5M) and \textit{SERICS-SuReCare} (€114.5M), where I develop MLOps practices and technical solutions for deploying AI models in production environments. I am also leading a joint NSF-MUR research proposal (\textit{DisTrac}, €215K) investigating AI-powered analysis of developer behavior in open source projects.
My track record includes successful coordination of international research projects, including leading the Italian unit of \textit{PRONEM} (Brazilian Ministry of Education) on natural language processing for global software development. 
Additionally, as co-founder of PeoplewareAI, a university spin-off, I bridge academic research and industry applications by developing commercial AI solutions for software engineering challenges. The company has successfully developed commercial AI solutions for behavioral analysis and MLOps and is currently participating in a regional initiaive (\textit{ARIANNA}, €1M funding proposal). 

\textbf{Proposed Research Program}. 
I look forward to establishing a research program that directly aligns with HPI's vision of advancing AI-driven software development while ensuring a human-centered approach:

\begin{itemize}
\item \textbf{Human-AI Collaboration in Software Engineering:} 
Building on my work in human factors and empirical software engineering, I will investigate the emerging paradigm where development teams orchestrate multiple specialized AI agents, particularly within enterprise software environments. This research will establish metrics for evaluating effectiveness in hybrid human-AI environments, develop frameworks for coordinating AI agents in complex tasks (e.g., code review, requirements engineering, and automated code generation), and create guidelines for integrating AI-based tools into enterprise development workflows while preserving meaningful human agency. A key focus will be understanding how teams can effectively direct multiple AI agents while maintaining coherent software architecture and ensuring business process integrity across sites.
\item \textbf{Quality Assurance for AI-Assisted Development:} 
Drawing from my experience in empirical validation and mining software repositories, I will develop novel verification techniques specifically designed for enterprise-scale AI-generated code and documentation. 
This research stream will create automated testing strategies for hybrid human-AI development environments, advance AI-based test case generation methods tailored to business-critical applications, and establish metrics to assess the reliability of AI-assisted software development in production environments. These insights will help establish rigorous quality standards for integrating AI tools into enterprise systems. Special attention will be paid to MLOps practices and tools that ensure consistent quality across the AI lifecycle in large-scale business environments.
\item \textbf{AI-Driven Software Process Evolution:} 
Leveraging my background in software engineering and industry collaboration, I will investigate how AI technologies transform traditional enterprise development methodologies. This research will focus on developing intelligent code management systems that connect repository data with enterprise AI tools, optimizing development workflows through automated code summarization, review, and fixing at scale, and creating process models that balance automation with human oversight in business-critical systems. A key emphasis will be on ensuring these processes support the development of robust, enterprise-ready AI applications.
\end{itemize}

\textbf{Institutional Fit}. 
The unique position of HPI at the intersection of academic research and industry collaboration provides an ideal environment for my research agenda. 
The emphasis on high standards in both research and teaching at the institute aligns with my academic approach.
The strong connections to industry at HPI align also with my experience in translating research into practical applications. 
Overall, this position at HPI represents an exciting opportunity to shape and advance the field of AI-driven software engineering while also contributing to the tradition of excellence at the institute.

I look forward to discussing how my expertise and vision align with HPI's goals in more detail.