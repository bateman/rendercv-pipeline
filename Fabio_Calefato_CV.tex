\documentclass[10pt, a4paper]{article}

% Packages:
\usepackage[
    ignoreheadfoot, % set margins without considering header and footer
    top=2 cm, % seperation between body and page edge from the top
    bottom=2 cm, % seperation between body and page edge from the bottom
    left=2 cm, % seperation between body and page edge from the left
    right=2 cm, % seperation between body and page edge from the right
    footskip=1.0 cm, % seperation between body and footer
    % showframe % for debugging 
]{geometry} % for adjusting page geometry
\usepackage{titlesec} % for customizing section titles
\usepackage{tabularx} % for making tables with fixed width columns
\usepackage{array} % tabularx requires this
\usepackage[dvipsnames]{xcolor} % for coloring text
\definecolor{primaryColor}{RGB}{0, 0, 0} % define primary color
\usepackage{enumitem} % for customizing lists
\usepackage{fontawesome5} % for using icons
\usepackage{amsmath} % for math
\usepackage[
    pdftitle={Fabio Calefato's CV},
    pdfauthor={Fabio Calefato},
    pdfcreator={LaTeX with RenderCV},
    colorlinks=true,
    urlcolor=primaryColor
]{hyperref} % for links, metadata and bookmarks
\usepackage[pscoord]{eso-pic} % for floating text on the page
\usepackage{calc} % for calculating lengths
\usepackage{bookmark} % for bookmarks
\usepackage{lastpage} % for getting the total number of pages
\usepackage{changepage} % for one column entries (adjustwidth environment)
\usepackage{paracol} % for two and three column entries
\usepackage{ifthen} % for conditional statements
\usepackage{needspace} % for avoiding page brake right after the section title
\usepackage{iftex} % check if engine is pdflatex, xetex or luatex
% Ensure that generate pdf is machine readable/ATS parsable:
\ifPDFTeX
    \input{glyphtounicode}
    \pdfgentounicode=1
    \usepackage[T1]{fontenc}
    \usepackage[utf8]{inputenc}
    \usepackage{lmodern}
\fi
\usepackage{charter}
\usepackage{graphicx} % for the profile picture

% Some settings:
\usepackage[none]{hyphenat}
\sloppy
\AtBeginEnvironment{adjustwidth}{\partopsep0pt} % remove space before adjustwidth environment
\pagestyle{empty} % no header or footer
\setcounter{secnumdepth}{0} % no section numbering
\setlength{\parindent}{0pt} % no indentation
\setlength{\topskip}{0pt} % no top skip
\setlength{\columnsep}{0.15cm} % set column seperation
\pagenumbering{gobble} % no page numbering

\titleformat{\section}{\needspace{4\baselineskip}\bfseries\large}{}{0pt}{}[\vspace{1pt}\titlerule]

\titlespacing{\section}{
    % left space:
    -1pt
}{
    % top space:
    0.3 cm
}{
    % bottom space:
    0.2 cm
} % section title spacing

\renewcommand\labelitemi{$\vcenter{\hbox{\small$\bullet$}}$} % custom bullet points
\newenvironment{highlights}{
    \begin{itemize}[
        topsep=0.10 cm,
        parsep=0.10 cm,
        partopsep=0pt,
        itemsep=0pt,
        leftmargin=0 cm + 10pt
    ]
}{
    \end{itemize}
} % new environment for highlights


\newenvironment{highlightsforbulletentries}{
    \begin{itemize}[
        topsep=0.10 cm,
        parsep=0.10 cm,
        partopsep=0pt,
        itemsep=0pt,
        leftmargin=10pt
    ]
}{
    \end{itemize}
} % new environment for highlights for bullet entries

\newenvironment{onecolentry}{
    \begin{adjustwidth}{
        0 cm + 0.00001 cm
    }{
        0 cm + 0.00001 cm
    }
}{
    \end{adjustwidth}
} % new environment for one column entries

\newenvironment{twocolentry}[2][]{
    \onecolentry
    \def\secondColumn{#2}
    \setcolumnwidth{\fill, 4.5 cm}
    \begin{paracol}{2}
}{
    \switchcolumn \raggedleft \secondColumn
    \end{paracol}
    \endonecolentry
} % new environment for two column entries

\newenvironment{threecolentry}[3][]{
    \onecolentry
    \def\thirdColumn{#3}
    \setcolumnwidth{, \fill, 4.5 cm}
    \begin{paracol}{3}
    {\raggedright #2} \switchcolumn
}{
    \switchcolumn \raggedleft \thirdColumn
    \end{paracol}
    \endonecolentry
} % new environment for three column entries

\newenvironment{header}{
    \setlength{\topsep}{0pt}\par\kern\topsep\centering\linespread{1.5}
}{
    \par\kern\topsep
} % new environment for the header

\newcommand{\placelastupdatedtext}{% \placetextbox{<horizontal pos>}{<vertical pos>}{<stuff>}
  \AddToShipoutPictureFG*{% Add <stuff> to current page foreground
    \put(
        \LenToUnit{\paperwidth-2 cm-0 cm+0.05cm},
        \LenToUnit{\paperheight-1.0 cm}
    ){\vtop{{\null}\makebox[0pt][c]{
        \small\color{gray}\textit{Updated$:$ Jan 2025}\hspace{\widthof{Updated$:$ Jan 2025}}
    }}}%
  }%
}%

\newenvironment{twocolheadercontainer}[2][]{
    \onecolentry
    \def\secondColumn{#2}
    \setcolumnwidth{\fill, 0.2\linewidth}
    \begin{paracol}{2}
}{
    \switchcolumn \raggedleft \secondColumn
    \end{paracol}
    \endonecolentry
} % new environment for headers with profile pictures

% save the original href command in a new command:
\let\hrefWithoutArrow\href

% new command for external links:
\renewcommand{\href}[2]{\hrefWithoutArrow{#1}{\ifthenelse{\equal{#2}{}}{ }{#2 }\raisebox{.15ex}{\footnotesize \faExternalLink*}}}


\begin{document}
    \newcommand{\AND}{\unskip
        \cleaders\copy\ANDbox\hskip\wd\ANDbox
        \ignorespaces
    }
    \newsavebox\ANDbox
    \sbox\ANDbox{$|$}

    \placelastupdatedtext

    \begin{header}
        \fontsize{25 pt}{25 pt}\selectfont Fabio Calefato

        \vspace{5 pt}

        \normalsize
        \mbox{{\footnotesize\faMapMarker*}\hspace*{0.13cm}Bari, Italy}%
        \kern 5.0 pt%
        \AND%
        \kern 5.0 pt%
        \mbox{\hrefWithoutArrow{mailto:fabio.calefato@uniba.it}{{\footnotesize\faEnvelope[regular]}\hspace*{0.13cm}fabio.calefato@uniba.it}}%
        \kern 5.0 pt%
        \AND%
        \kern 5.0 pt%
        \mbox{\hrefWithoutArrow{tel:+39-080-571-2213}{{\footnotesize\faPhone*}\hspace*{0.13cm}+39 080 571 2213}}%
        \kern 5.0 pt%
        \AND%
        \kern 5.0 pt%
        \mbox{\hrefWithoutArrow{https://collab.di.uniba.it/fabio}{{\footnotesize\faLink}\hspace*{0.13cm}collab.di.uniba.it/fabio}}%
        \kern 5.0 pt%
        \AND%
        \kern 5.0 pt%
        \mbox{\hrefWithoutArrow{https://orcid.org/0000-0003-2654-1588}{{\footnotesize\faOrcid}\hspace*{0.13cm}0000-0003-2654-1588}}%
        \kern 5.0 pt%
        \AND%
        \kern 5.0 pt%
        \mbox{\hrefWithoutArrow{https://scholar.google.com/citations?user=n_XWRkoAAAAJ}{{\footnotesize\faGraduationCap}\hspace*{0.13cm}Google Scholar}}%
        \kern 5.0 pt%
        \AND%
        \kern 5.0 pt%
        \mbox{\hrefWithoutArrow{https://github.com/bateman}{{\footnotesize\faGithub}\hspace*{0.13cm}bateman}}%
    \end{header}

    \vspace{15 pt - 0.3 cm}


    \section{Research Experience}



        
        \begin{twocolentry}{
            Nov 2022 – present
        }
            \textbf{Associate Professor}, University of Bari, Dept. of Computer Science -- Bari, Italy\end{twocolentry}

        \vspace{0.10 cm}
        \begin{onecolentry}
            \begin{highlights}
                \item Holds the national habilitation as Full Professor since Dec 2023
            \end{highlights}
        \end{onecolentry}


        \vspace{0.2 cm}

        \begin{twocolentry}{
            Feb 2021 – present
        }
            \textbf{Co-founder and CEO}, PeoplewareAI s.r.l. -- Bari, Italy\end{twocolentry}

        \vspace{0.10 cm}
        \begin{onecolentry}
            \begin{highlights}
                \item \href{https://peopleware.ai}{PeoplewareAI} is a spin-off company of the University of Bari, founded to transfer the results of research in the field of AI and software engineering to the market
                \item Direct R\&D and product development in human-centered AI applications, including emotion/sentiment analysis tools for technical communication and MLOps automation pipelines for healthcare systems
                \item Foster industry-academia collaboration through consulting services and technology transfer projects
            \end{highlights}
        \end{onecolentry}


        \vspace{0.2 cm}

        \begin{twocolentry}{
            Nov 2019 – Nov 2022
        }
            \textbf{Tenure-track Assistant Professor}, University of Bari, Dept. of Computer Science -- Bari, Italy\end{twocolentry}



        \vspace{0.2 cm}

        \begin{twocolentry}{
            Nov 2015 – Nov 2019
        }
            \textbf{Untenured Assistant Professor}, University of Bari, Jonian Dept. -- Taranto, Italy\end{twocolentry}



        \vspace{0.2 cm}

        \begin{twocolentry}{
            July 2013 – July 2015
        }
            \textbf{Postdoctoral Research Fellow}, University of Bari, Dept. of Computer Science -- Bari, Italy\end{twocolentry}



        \vspace{0.2 cm}

        \begin{twocolentry}{
            Apr 2010 – Mar 2013
        }
            \textbf{Postdoctoral Research Fellow}, University of Bari, Dept. of Computer Science -- Bari, Italy\end{twocolentry}



        \vspace{0.2 cm}

        \begin{twocolentry}{
            Apr 2007 – Nov 2008
        }
            \textbf{Postdoctoral Research Fellow}, University of Bari, Dept. of Computer Science -- Bari, Italy\end{twocolentry}




    
    \section{Education}



        
        \begin{twocolentry}{
            July 2008 – June 2009
        }
            \textbf{University of Bari, Italy}, \textbf{Inter-University Specialization School for Secondary Education} in \textbf{Physics, Computer Science, and Mathematics}\end{twocolentry}

        \vspace{0.10 cm}
        \begin{onecolentry}
            \begin{highlights}
                \item \textit{Graduation grade: 42/42}
            \end{highlights}
        \end{onecolentry}


        \vspace{0.2 cm}

        \begin{twocolentry}{
            Jan 2004 – May 2007
        }
            \textbf{University of Bari, Italy}, \textbf{PhD} in \textbf{Computer Science}\end{twocolentry}

        \vspace{0.10 cm}
        \begin{onecolentry}
            \begin{highlights}
                \item \textit{Thesis: ``Supporting Synchronous Communication in Distributed Software Teams''}
                \item \textit{Thesis listed among the \href{https://www.sigsoft.org/dissertations.html}{ACM SIGSOFT selected Ph.D. Dissertations in the Area of Software Engineering}}
                \item \textit{Supervisor: Prof. Filippo Lanubile (Uniba)}
                \item \textit{Co-supervisor: Prof. Daniela Damian (UVic)}
            \end{highlights}
        \end{onecolentry}


        \vspace{0.2 cm}

        \begin{twocolentry}{
            Sept 1996 – Oct 2002
        }
            \textbf{University of Bari, Italy}, \textbf{MSc} in \textbf{Computer Science}\end{twocolentry}

        \vspace{0.10 cm}
        \begin{onecolentry}
            \begin{highlights}
                \item \textit{Thesis: ``P2P Conferences in JXTA''}
                \item \textit{Graduation grade: 110/110 with honors}
                \item \textit{Supervisor: Prof. Filippo Lanubile (Uniba)}
            \end{highlights}
        \end{onecolentry}



    
    \section{Research Activity}



        
        \begin{onecolentry}
            My research primarily focuses on the intersection of Software Engineering and AI/ML, while also encompassing human factors in software development and globally distributed software engineering. Throughout my career, I have maintained a strong focus on empirical validation of research findings, conducting controlled experiments, case studies, and mining open-source software repositories. My work has consistently appeared in top-tier venues and has influenced both academic research and industry practices in software engineering. Below is a detailed description of some of my research activities, organized by recency and impact.
        \end{onecolentry}

        \vspace{0.2 cm}

        \begin{onecolentry}
            \textbf{Generative AI for software engineering research and practices}: In collaboration with leading international researchers, I am advancing three active workstreams that emerged from my participation in the 2023 and 2024 editions of the \href{https://www.danielrusso.org/copenhagen-symposium-human-centered-ai-software-engineering}{\textit{Copenhagen Symposium on Human-Centered Software Engineering AI}}. The first workstream comprises an observational study mining self-admitted mentions of LLMs usage in open-source projects. Together, we examined how developers integrate AI assistants into their workflows across development tasks, content types, and usage purposes. Our study analyzed over 250,000 open-source repositories, identifying patterns in AI tool adoption and their impact on project metrics. An article describing this collaborative work is currently under review at IEEE TSE. The second workstream aims to establish a comprehensive set of guidelines for conducting experiments with LLMs in software engineering research. Our joint initiative addresses the challenges of achieving reproducible results with LLMs by tackling their unique characteristics that affect study validity and reproducibility, providing researchers with concrete protocols for empirical evaluations. The third workstream investigates the integration of AI in software engineering research methodologies. Together, we examine how Generative AI tools can support various research tasks including qualitative analysis, systematic literature reviews, and human studies design. Our collaborative work explores both opportunities and risks of AI adoption in SE research, gathering perspectives from researchers about the changing landscape of empirical software engineering methods.
        \end{onecolentry}

        \vspace{0.2 cm}

        \begin{onecolentry}
            \textbf{Software engineering for AI-enabled systems}: One of my recent research interests is focused on improving the development workflows of AI/ML-based systems through empirical studies and tool development. I conducted a comprehensive review of industry-leading AutoML tools to analyze their benefits and limitations in software engineering contexts (IST 2025). I have contributed to understanding MLOps practices by analyzing adoption patterns in open-source projects on GitHub (ESEM 2022). This work revealed key challenges in transitioning ML models from experimentation to production, leading to the development of an MLOps solution framework applied in healthcare contexts (CAIN 2022). Finally, I have established best practices for collaborative development of AI systems using computational notebooks (CSCW 2021).
        \end{onecolentry}

        \vspace{0.2 cm}

        \begin{onecolentry}
            \textbf{AI Safety and Regulatory Compliance in Healthcare Systems}: One of my current research interests focuses on developing frameworks and methodologies for ensuring continuous compliance and safety of AI systems in regulated healthcare environments. This research addresses the challenge of maintaining regulatory adherence while enabling continuous learning in medical AI applications. Working with medical professionals and life science researchers, I am developing an extended MLOps framework that integrates automated compliance verification, monitoring, and ethical oversight throughout the AI system lifecycle. The framework introduces systematic approaches for bias detection, fairness assessment, and performance monitoring across demographic groups, bridging the gap between responsible AI principles and clinical implementation requirements. This work has fostered collaborations with healthcare institutions and secured funding through national initiatives, including the DARE project (€130.5M) for digital preventive healthcare solutions.
        \end{onecolentry}

        \vspace{0.2 cm}

        \begin{onecolentry}
            \textbf{Industry-based research on the state of software engineering practices}: I have participated in and continue to contribute to several industry-based global surveys to understand software engineering practices. The \href{https://helenastudy.wordpress.com/helena-team}{\textit{HELENA}} (Hybrid dEveLopmENt Approaches in software systems development) project has identified key characteristics of hybrid development approaches through analysis of 1,000+ developers across 50 countries since 2016. Our findings on agile process adoption patterns appeared in IEEE TSE (2021), significantly impacting our understanding of modern development methodologies. The \href{http://www.napire.org}{\textit{NaPiRE}} project (Naming the Pain in Requirements Engineering) is a global survey initiative examining industrial practices and challenges in Requirements Engineering. Through biannual surveys, our large-scale academic collaboration develops a holistic theory of RE practices and problems, producing insights that guide problem-driven research. The \textit{Evolution of Post-Pandemic Work Policies} project analyzes hybrid and remote work policies across companies worldwide through global surveys and academic collaboration. Our research provides evidence-based insights into emerging work patterns, revealing challenges and adaptations in post-pandemic work environments and helping organizations optimize their hybrid workplace policies.
        \end{onecolentry}

        \vspace{0.2 cm}

        \begin{onecolentry}
            \textbf{Human factors in software engineering}: In my research I have extensively investigated how human factors such, as personality traits, emotions, and social dynamics, influence software development processes, leveraging AI/ML techniques for analysis across various developer platforms and communication channels. In technical Q\&A platforms like Stack Overflow, I have conducted comprehensive studies analyzing both technical aspects (such as community guidelines for effective questions) and social factors affecting answer success rates (MSR 2015, ESEM 2016, IST 2018, EMSE 2019). This work has led to the creation of gold standards for sentiment analysis (MSR 2018) and the development of ML-based methods to detect emotions and sentiment polarity in technical communication (IEEE Software 2020). Additionally, I have conducted cross-platform evaluations of sentiment analysis tools (MSR 2020) and performed extended replications to assess how the choice of sentiment analysis tools influences the validity of empirical studies (EMSE 2021). Beyond sentiment analysis, I have investigated how developer personalities influence collaboration in large software ecosystems like Apache (ICGSE 2018, IST 2019), with particular attention to how traits like agreeableness impact code review activities and pull request acceptance (ICGSE 2017). My work has shed light on the need for developing specialized tools for automatic personality detection from text in technical contexts (TOSEM 2021). With this line of research, I have demonstrated the critical importance of domain-specific approaches when analyzing developer communications, showing the limitations of general-purpose personality and sentiment analysis tools in software engineering contexts. Finally, I have also studied retention and disengagement factors of Open Source Software community participants, defining and validating a theoretical model of the activity rhythm of open-source project developers (SOHEAL 2019, EMSE 2022).
        \end{onecolentry}

        \vspace{0.2 cm}

        \begin{onecolentry}
            \textbf{Global software engineering}: My research has addressed the challenges of software development distributed on a global scale. Key contributions include theoretical and empirical work on trust-building mechanisms and social awareness in virtual teams (CSCW 2013, CHASE 2012, IEEE Software 2013). I pioneered SocialCDE, a social awareness tool for fostering trust in distributed teams (ESEC/FSE 2013), which was awarded the \textit{2011 Microsoft Software Engineering Innovation Award}; this work demonstrated how social awareness tools can increase trust and improve coordination in global teams. I also made significant advances in communication barriers, developing and evaluating eConference, a real-time ML-based translation tool (ICGSE 2010-11, ESEM 2012, ESEM 2014, ESE 2016) that showed promising efficiency gains while identifying important trade-offs in distributed development activities; the tool was awarded the \textit{2006 Eclipse Innovation Award} by IBM. Additional contributions include an industrial action research study on communication tools in distributed agile teams (ICGSE 2020). My expertise in this domain is reflected in my service as General Chair for ICGSE 2019 and my role as Guest Editor for JSS special issue on Global Software Engineering (JSS 2021).
        \end{onecolentry}


    
    \section{Awards}



        
        \begin{onecolentry}
            \textbf{Best Paper Award:} \textit{14th Int'l Conf. on Global Software Engineering (ICGSE'19), Montreal, Canada}
        \end{onecolentry}

        \vspace{0.2 cm}

        \begin{onecolentry}
            \textbf{\href{https://www.anvur.it/wp-content/uploads/2018/05/Beneficiari_FFABR_Ricercatori.pdf}{FABBR 2017 Award}:} \textit{Winner of the national selection procedure `Fondo per il finanziamento delle attività base di ricerca (FFABR) 2017,' established by the Italian Ministry of University and Research (MUR) and intended for the annual funding of basic research activities of associate professors and researchers}
        \end{onecolentry}


    
    \section{Bibliometrics}



        
        \begin{onecolentry}
            \textbf{Google Scholar:} \textit{h}-index 30, 3,150+ citations
        \end{onecolentry}

        \vspace{0.2 cm}

        \begin{onecolentry}
            \textbf{Scopus:} \textit{h}-index 21, 1,500+ citations
        \end{onecolentry}


    
    \section{Selected Publications}



        
        \begin{samepage}
            \begin{twocolentry}{
                2025 - under review
            }
                \textbf{On the Self-admitted Use of LLMs in Open Source Software Projects}
            \end{twocolentry}

            \vspace{0.10 cm}
            
            \begin{onecolentry}
                \mbox{T. Xiao}, \mbox{Y. Fan}, \mbox{\textbf{\textit{F. Calefato}}}, \mbox{R. Kula}, \mbox{C. Treude}, \mbox{H. Hata}, \mbox{S. Baltes}

                \vspace{0.10 cm}
                
        \textit{IEEE Transactions on Software Engineering}\end{onecolentry}
        \end{samepage}

        \vspace{0.2 cm}

        \begin{samepage}
            \begin{twocolentry}{
                2025
            }
                \textbf{A multivocal literature review on the benefits and limitations of industry-leading AutoML tools}
            \end{twocolentry}

            \vspace{0.10 cm}
            
            \begin{onecolentry}
                \mbox{L. Quaranta}, \mbox{K. Azevedo}, \mbox{\textbf{\textit{F. Calefato}}}, \mbox{M. Kalinowski}

                \vspace{0.10 cm}
                
        \textit{Inf. Softw. Technol., vol. 178}, doi: \href{https://doi.org/10.1016/J.INFSOF.2024.107608}{10.1016/J.INFSOF.2024.107608}, rank: SJR Q1
        \end{onecolentry}
        \end{samepage}

        \vspace{0.2 cm}

        \begin{samepage}
            \begin{twocolentry}{
                2024
            }
                \textbf{Generative AI in Software Engineering Must Be Human-Centered: The Copenhagen Manifesto}
            \end{twocolentry}

            \vspace{0.10 cm}
            
            \begin{onecolentry}
                \mbox{D. Russo}, \mbox{S. Baltes}, \mbox{N. Berkel}, \mbox{P. Avgeriou}, \mbox{\textbf{\textit{F. Calefato}}}, \mbox{B. Cabrero-Daniel}, \mbox{G. Catolino}, \mbox{J. Cito}, \mbox{N. Ernst}, \mbox{T. Fritz}, \mbox{H. Hata}, \mbox{R. Holmes}, \mbox{M. Izadi}, \mbox{F. Khomh}, \mbox{M. Kjærgaard}, \mbox{G. Liebel}, \mbox{A. Lluch-Lafuente}, \mbox{S. Lambiase}, \mbox{W. Maalej}, \mbox{G. Murphy}, \mbox{N. Moe}, \mbox{G. O'Brien}, \mbox{E. Paja}, \mbox{M. Pezzè}, \mbox{J. Persson}, \mbox{R. Prikladnicki}, \mbox{P. Ralph}, \mbox{M. Robillard}, \mbox{T. Silva}, \mbox{K.J. Stol}, \mbox{M.A. Storey}, \mbox{V. Stray}, \mbox{P. Tell}, \mbox{C. Treude}, \mbox{B. Vasilescu}

                \vspace{0.10 cm}
                
        \textit{J. Syst. Softw., vol. 216}, doi: \href{https://doi.org/10.1016/J.JSS.2024.112115}{10.1016/J.JSS.2024.112115}, rank: SJR Q1
        \end{onecolentry}
        \end{samepage}

        \vspace{0.2 cm}

        \begin{samepage}
            \begin{twocolentry}{
                2023
            }
                \textbf{Assessing the Use of AutoML for Data-Driven Software Engineering}
            \end{twocolentry}

            \vspace{0.10 cm}
            
            \begin{onecolentry}
                \mbox{\textbf{\textit{F. Calefato}}}, \mbox{L. Quaranta}, \mbox{F. Lanubile}, \mbox{M. Kalinowski}

                \vspace{0.10 cm}
                
        \textit{ACM/IEEE International Symposium on Empirical Software Engineering and Measurement, ESEM 2023, New Orleans, LA, USA, October 26-27, 2023}, doi: \href{https://doi.org/10.1109/ESEM56168.2023.10304796}{10.1109/ESEM56168.2023.10304796}, rank: ICORE A
        \end{onecolentry}
        \end{samepage}

        \vspace{0.2 cm}

        \begin{samepage}
            \begin{twocolentry}{
                2021
            }
                \textbf{Towards Productizing AI/ML Models: An Industry Perspective from Data Scientists}
            \end{twocolentry}

            \vspace{0.10 cm}
            
            \begin{onecolentry}
                \mbox{F. Lanubile}, \mbox{\textbf{\textit{F. Calefato}}}, \mbox{L. Quaranta}, \mbox{M. Amoruso}, \mbox{F. Fumarola}, \mbox{M. Filannino}

                \vspace{0.10 cm}
                
        \textit{1st IEEE/ACM Workshop on AI Engineering - Software Engineering for AI, WAIN@ICSE 2021, Madrid, Spain, May 30-31, 2021}, doi: \href{https://doi.org/10.1109/WAIN52551.2021.00027}{10.1109/WAIN52551.2021.00027}\end{onecolentry}
        \end{samepage}

        \vspace{0.2 cm}

        \begin{samepage}
            \begin{twocolentry}{
                2020
            }
                \textbf{Love, Joy, Anger, Sadness, Fear, and Surprise - SE Needs Special Kinds of AI: A Case Study on Text Mining and SE}
            \end{twocolentry}

            \vspace{0.10 cm}
            
            \begin{onecolentry}
                \mbox{N. Novielli}, \mbox{\textbf{\textit{F. Calefato}}}, \mbox{F. Lanubile}

                \vspace{0.10 cm}
                
        \textit{IEEE Softw., vol. 37, no. 3}, doi: \href{https://doi.org/10.1109/MS.2020.2968557}{10.1109/MS.2020.2968557}, rank: SJR Q2
        \end{onecolentry}
        \end{samepage}


    
    \section{Ph.D. Students Supervision}



        
        \begin{twocolentry}{
            July 2023 – present
        }
            \textbf{Kelly Azevedo Borges Leal} -- Pontifical Catholic University of Rio de Janeiro (PUC-Rio), Brazil\end{twocolentry}

        \vspace{0.10 cm}
        \begin{onecolentry}
            \begin{highlights}
                \item \textit{Topic: AutoML for Software Engineering in Practice}
                \item \textit{Co-supervised with prof. Co-supervised with prof. Marcos Kalinowski}
            \end{highlights}
        \end{onecolentry}


        \vspace{0.2 cm}

        \begin{twocolentry}{
            Nov 2022
        }
            \textbf{Luigi Quaranta} -- University of Bari, Italy\end{twocolentry}

        \vspace{0.10 cm}
        \begin{onecolentry}
            \begin{highlights}
                \item \textit{Thesis: ``Collaboration Around Computational Notebooks To Build AI-enabled Systems''}
                \item \textit{Co-supervised with prof. Filippo Lanubile}
            \end{highlights}
        \end{onecolentry}


        \vspace{0.2 cm}

        \begin{twocolentry}{
            Feb 2020
        }
            \textbf{Giuseppe Iaffaldano} -- University of Bari, Italy\end{twocolentry}

        \vspace{0.10 cm}
        \begin{onecolentry}
            \begin{highlights}
                \item \textit{Thesis: ``Investigating the Dynamics of Online Creative Communities''}
                \item \textit{Co-supervised with prof. Filippo Lanubile}
            \end{highlights}
        \end{onecolentry}



    
    \section{Teaching}



        
        \begin{twocolentry}{
            2022 – 2025
        }
            \textbf{Reti di Calcolatori (Computer Networks) [6 ECTS]}, BSc in Computer Science and Software Production Technologies (ITPS), 2nd year, University of Bari, Dept. of Computer Science -- Bari, Italy\end{twocolentry}

        \vspace{0.10 cm}
        \begin{onecolentry}
            \begin{highlights}
                \item \textit{Teaching evaluation scores:  2023-25: N/A; 2022-23: 95.52\%}
            \end{highlights}
        \end{onecolentry}


        \vspace{0.2 cm}

        \begin{twocolentry}{
            2021 – 2024
        }
            \textbf{Social Computing [6 ECTS]}, MSc in Computer Science, 1st year, University of Bari, Dept. of Computer Science -- Bari, Italy\end{twocolentry}

        \vspace{0.10 cm}
        \begin{onecolentry}
            \begin{highlights}
                \item \textit{Teaching evaluation scores: 2023-24: N/A; 2022-23: 100\%; 2021-22: 100\%}
            \end{highlights}
        \end{onecolentry}


        \vspace{0.2 cm}

        \begin{twocolentry}{
            2022 – 2023
        }
            \textbf{Software Solutions for Reproducible Experiments [2 ECTS]}, Ph.D. program in Computer Science and Mathematics (XXXVIII cycle), University of Bari, Dept. of Computer Science -- Bari, Italy\end{twocolentry}



        \vspace{0.2 cm}

        \begin{twocolentry}{
            2022 – 2023
        }
            \textbf{Social Computing [5 ECTS]}, MSc in Computer Science and Engineering, University of Oulu -- Oulu, Finland\end{twocolentry}

        \vspace{0.10 cm}
        \begin{onecolentry}
            \begin{highlights}
                \item \textit{Course taught as the winner of an EU-funded grant `Erasmus+ TUCEP Selection Call Mobility for teaching staff' (2022-23), for university professors and researchers mobility}
            \end{highlights}
        \end{onecolentry}


        \vspace{0.2 cm}

        \begin{twocolentry}{
            2021 – 2022
        }
            \textbf{Reti di Calcolatori (Computer Networks) [9 ECTS]}, BSc in Computer Science, 3rd year in University of Bari, Dept. of Computer Science -- Bari, Italy\end{twocolentry}

        \vspace{0.10 cm}
        \begin{onecolentry}
            \begin{highlights}
                \item \textit{Teaching evaluation scores: 97.07\%}
            \end{highlights}
        \end{onecolentry}


        \vspace{0.2 cm}

        \begin{twocolentry}{
            2019 – 2021
        }
            \textbf{Reti di Calcolatori (Computer Networks) [6 ECTS]}, BSc in Computer Science and Software Production Technologies (ITPS), 2nd year, University of Bari, Dept. of Computer Science -- Bari, Italy\end{twocolentry}

        \vspace{0.10 cm}
        \begin{onecolentry}
            \begin{highlights}
                \item \textit{Teaching evaluation scores: 2020-21: 94.59\%; 2019-20: 89.74\%}
            \end{highlights}
        \end{onecolentry}


        \vspace{0.2 cm}

        \begin{twocolentry}{
            2018 – 2019
        }
            \textbf{Reti di Calcolatori (Computer Networks) [9 ECTS]}, BSc in Computer Science, 3rd year in University of Bari, Dept. of Computer Science -- Bari, Italy\end{twocolentry}

        \vspace{0.10 cm}
        \begin{onecolentry}
            \begin{highlights}
                \item \textit{Teaching evaluation scores: 93.4\%}
            \end{highlights}
        \end{onecolentry}


        \vspace{0.2 cm}

        \begin{twocolentry}{
            2017 – 2018
        }
            \textbf{Mining Socio-technical Repositories [3 ECTS]}, Ph.D. program in Computer Science and Mathematics (XXXIII cycle), University of Bari, Dept. of Computer Science -- Bari, Italy\end{twocolentry}



        \vspace{0.2 cm}

        \begin{twocolentry}{
            2017 – 2018
        }
            \textbf{IT Tools Supporting Legal and Economic Research: Blockchain for Tracking Production and Transportation Chains [2 ECTS]}, Ph.D. program in Rights, Economies, and Cultures of the Mediterranean (XXXIII cycle), University of Bari, Jonian Dept. (Law and Business School) -- Taranto, Italy\end{twocolentry}



        \vspace{0.2 cm}

        \begin{twocolentry}{
            2016 – 2017
        }
            \textbf{Classification Models in Software Engineering [2 ECTS]}, Ph.D. program in Computer Science, University of Oulu -- Oulu, Finland\end{twocolentry}

        \vspace{0.10 cm}
        \begin{onecolentry}
            \begin{highlights}
                \item \textit{Course taught as the winner of an EU-funded grant `Erasmus+ TUCEP Selection Call Mobility for teaching staff' (2015-16), for university professors and researchers mobility}
            \end{highlights}
        \end{onecolentry}


        \vspace{0.2 cm}

        \begin{twocolentry}{
            2016 – 2017
        }
            \textbf{Emotion Awareness in Social Computing [2 ECTS]}, Ph.D. program in Computer Science and Mathematics (XXXII cycle), University of Bari, Dept. of Computer Science -- Oulu, Finland\end{twocolentry}

        \vspace{0.10 cm}
        \begin{onecolentry}
            \begin{highlights}
                \item Co-taught with Prof. Nicole Novielli
            \end{highlights}
        \end{onecolentry}


        \vspace{0.2 cm}

        \begin{twocolentry}{
            2016 – 2017
        }
            \textbf{IT Tools Supporting Legal and Economic Research [2 ECTS]}, Ph.D. program in Rights, Economies, and Cultures of the Mediterranean (XXXII cycle), University of Bari, Jonian Dept. (Law and Business School) -- Taranto, Italy\end{twocolentry}



        \vspace{0.2 cm}

        \begin{twocolentry}{
            2015 – 2020
        }
            \textbf{Informatica (Computer Science) [9 ECTS]}, BSc in Maritime Sciences and Management (SGAM), 2nd year -- Italian Navy Petty Officers School (Mariscuola), Taranto, Italy, University of Bari, Dept. of Computer Science\end{twocolentry}

        \vspace{0.10 cm}
        \begin{onecolentry}
            \begin{highlights}
                \item \textit{Teaching evaluation scores: 2019-20: 95.2\%; 2018-19: 95.9\%; 2017-18: 97.4\%; 2016-17: 91.9\%; 2015-16: 83.5\%}
            \end{highlights}
        \end{onecolentry}


        \vspace{0.2 cm}

        \begin{twocolentry}{
            2015 – 2020
        }
            \textbf{Abilità Informatica (Computer Skills) [4 ECTS]}, Bachelor of Law, 1st year, University of Bari, Jonian Dept. (Law and Business School) -- Taranto, Italy\end{twocolentry}

        \vspace{0.10 cm}
        \begin{onecolentry}
            \begin{highlights}
                \item \textit{Teaching evaluation scores: 2019-20: 93.1\%; 2018-19: 95.9\%; 2017-18: 93.8\%; 2016-17: 89.8\%; 2015-16: 87.8\%}
            \end{highlights}
        \end{onecolentry}


        \vspace{0.2 cm}

        \begin{twocolentry}{
            2013 – 2015
        }
            \textbf{Laboratorio di Informatica (C Programming Lab) [9 ECTS]}, BSc in Computer Science and Software Production Technologies (ITPS), 1st year, University of Bari, Dept. of Computer Science -- Bari, Italy\end{twocolentry}

        \vspace{0.10 cm}
        \begin{onecolentry}
            \begin{highlights}
                \item \textit{Teaching evaluation scores: 2014-15: 88.3\%; 2013-14: 93.1\%}
            \end{highlights}
        \end{onecolentry}


        \vspace{0.2 cm}

        \begin{twocolentry}{
            2013 – 2014
        }
            \textbf{Linguaggi di Programmazione + Laboratorio (Programming Languages + Lab) [12 ECTS]}, BSc in Computer Science, 1st year, University of Bari, Dept. of Computer Science -- Brindisi, Italy\end{twocolentry}

        \vspace{0.10 cm}
        \begin{onecolentry}
            \begin{highlights}
                \item \textit{Teaching evaluation scores: 95.7\%}
            \end{highlights}
        \end{onecolentry}


        \vspace{0.2 cm}

        \begin{twocolentry}{
            2008 – 2013
        }
            \textbf{Laboratorio di Informatica (C Programming Lab) [5/6 ECTS]}, BSc in Computer Science, 1st year, University of Bari, Dept. of Computer Science -- Brindisi, Italy\end{twocolentry}

        \vspace{0.10 cm}
        \begin{onecolentry}
            \begin{highlights}
                \item \textit{Teaching evaluation scores: N/A}
            \end{highlights}
        \end{onecolentry}



    
    \section{Funded Research Projects}



        
        \begin{twocolentry}{
            Jan 2024 – Dec 2027
        }
            \textbf{\href{https://www.fondazionedare.it/en/}{DARE - Digital Lifelong Prevention}}\end{twocolentry}

        \vspace{0.10 cm}
        \begin{onecolentry}
            \begin{highlights}
                \item \textit{Funded by:} MUR - Piano nazionale per gli investimenti complementari al Piano nazionale di ripresa e resilienza
                \item \textit{Project funding:} € 130.456.001,02
                \item \textit{Partners:} University of Bologna (coordinator), Università Cattolica del Sacro Cuore, University of Bari, University of Palermo, University of Padova, University of Rome Tor Vergata, INFN, multiple research hospitals and healthcare companies
                \item \textit{Description:} National initiative creating and developing a connected and distributed knowledge community for digital preventive healthcare through research, innovation, and participation of multiple stakeholders. The project produces, collects, and systematizes multidisciplinary knowledge and solutions (technical, ethical-legal, and organizational) necessary to ensure digital prevention in Italy.
                \item \textit{Role:} Lead researcher in Spoke 1 focusing on MLOps and AI Engineering. Responsible for designing and developing technological solutions for secure deployment and monitoring of AI models in healthcare settings. Key contributions include defining MLOps practices for healthcare and conducting systematic reviews of security risks and best practices.
            \end{highlights}
        \end{onecolentry}


        \vspace{0.2 cm}

        \begin{twocolentry}{
            Jan 2023 – Dec 2025
        }
            \textbf{\href{https://serics.eu/en/}{SERICS - SEcurity and RIghts In the CyberSpace / Spoke 9: SuReCare}}\end{twocolentry}

        \vspace{0.10 cm}
        \begin{onecolentry}
            \begin{highlights}
                \item \textit{Funded by:} MUR - Piano nazionale di ripresa e resilienza
                \item \textit{Project funding:} € 114.499.997,53
                \item \textit{Partners:} Sapienza University of Rome (coordinator), University of Bari, University of Cagliari
                \item \textit{Description:} Research initiative investigating cybersecurity challenges in remote healthcare systems through novel methodological approaches and technical innovations. The project advances the state-of-the-art in three key areas: ecosystem security for remote medical devices, end-to-end data security for sensitive health information, and automated detection-response-prevention mechanisms for decentralized healthcare infrastructures.
                \item \textit{Role:} Work package leader for WP3 (Detection-Response-Prevention). Led research on integrating ML models in healthcare cybersecurity systems, focusing on developing effective approaches to control the transitioning of ML models into production. Defined quality assurance functions for both data and models after deployment in the remote healthcare domain.
            \end{highlights}
        \end{onecolentry}


        \vspace{0.2 cm}

        \begin{twocolentry}{
            Nov 2018 – July 2020
        }
            \textbf{C3 - Creative Cultural Collaboration}\end{twocolentry}

        \vspace{0.10 cm}
        \begin{onecolentry}
            \begin{highlights}
                \item \textit{Funded by:} POR Puglia, Axis I, Action 1.6, INNONETWORK Program (FESR-FSE 2014-2020)
                \item \textit{Project funding:} € 383.852,47
                \item \textit{Partners:} AI2 srl (project coordinator), DABIMUS srl, Quorum Italia srl, Marshmallow Games srl, University of Bari
                \item \textit{Description:} Regional project that investigated methodological and technological frameworks for creating computational artifacts to enable scalable production processes and enhance collaboration in large, distributed multidisciplinary teams.
                \item \textit{Role:} Workpackage Leader for the University of Bari unit. Led the Operational Unit responsible for the collaborative platform workpackage (OR3), which included conducting a state-of-the-art analysis of open-source co-working solutions, platform design, and ongoing maintenance and evolution activities. The University of Bari unit was coordinated by Prof. Filippo Lanubile.
            \end{highlights}
        \end{onecolentry}


        \vspace{0.2 cm}

        \begin{twocolentry}{
            Sept 2015 – Sept 2017
        }
            \textbf{OpEn - Open up Entrepreneurship}\end{twocolentry}

        \vspace{0.10 cm}
        \begin{onecolentry}
            \begin{highlights}
                \item \textit{Funded by:} EU Erasmus+ Program (2015-1-EL-KA202-014168)
                \item \textit{Project funding:} € 229.193,00
                \item \textit{Partners:} Small Enterprises’ Institute of the Hellenic Confederation of Professionals, Craftsmen, and Merchants (IME GSEVEE, project coordinator), The European Business and Innovation Centre of Burgos (CEEI-Burgos), University of Bari, University of Patras, Manchester Metropolitan University
                \item \textit{Description:} International project that aimed at addressing the critical gap between entrepreneurs' digital technology needs and their e-business capabilities in global markets. The project developed multidisciplinary open educational resources to foster entrepreneurial mindset and digital business competencies for existing and prospective entrepreneurs Project coordination was led by Vassilis Siomadis (IME GSEVEE).
                \item \textit{Role:} Workpackage Leader for the University of Bari unit. Led operational teams in two key workpackages: designed thematic unit structure and e-class outline for the e-learning portal (WP2: E-module Service Design and Setup), and developed educational content focusing on web-based knowledge management tools (WP3: OpEn Educational Material). The University of Bari unit was coordinated by Prof. Filippo Lanubile.
            \end{highlights}
        \end{onecolentry}


        \vspace{0.2 cm}

        \begin{twocolentry}{
            Sept 2013 – Sept 2015
        }
            \textbf{VINCENTE - A Virtual Collective Intelligence Environment to Develop Sustainable Technology Entrepreneurship Ecosystems}\end{twocolentry}

        \vspace{0.10 cm}
        \begin{onecolentry}
            \begin{highlights}
                \item \textit{Funded by:} MUR PONREC 2007-2013 (PON02\_00563\_3470993)
                \item \textit{Project funding:} € 8.548.807,40
                \item \textit{Partners:} Engineering Computer Engineering, Exprivia SpA, ST Microelectronics, San Raffaele Hospital,  AVIO, Alenia, Tozzi Renewable Energy, CNR, University of Salento, University of Bari, Politecnico di Bari, DHITECH
                \item \textit{Description:} National project that aimed at strengthening research cooperation networks between academia and industry to enhance competitiveness and economic growth, while promoting the adoption of advanced technologies and services.
                \item \textit{Role:} Work package leader. Led the work unit on 'Innovative Models and Components for Social Networking' (OR6), focusing on developing interaction frameworks for collaborative environments. Specifically designed computer-mediated communication frameworks and selected appropriate communication tools for the VINCENTE virtual environment. The University of Bari unit was coordinated by Prof. Donato Malerba.
            \end{highlights}
        \end{onecolentry}


        \vspace{0.2 cm}

        \begin{twocolentry}{
            Mar 2012 – Mar 2014
        }
            \textbf{PRONEM - Natural Language Processing for Global Software Development}\end{twocolentry}

        \vspace{0.10 cm}
        \begin{onecolentry}
            \begin{highlights}
                \item \textit{Funded by:} Brazilian Ministry of Education (PRONEM/FAPERGS/CNPQ 03/2011)
                \item \textit{Unit funding: R\$ 249,600.00}
                \item \textit{Partners:} Pontifícia Universidade Católica do Rio Grande do Sul (PUCRS, project coordinator), Federal University of Rio Grande do Sul (UFRGS), University of Bari
                \item \textit{Description:} International project that aimed at investigating machine translation's impact on Brazil's global software development capabilities, addressing the critical challenge that only 10\% of professional developers were proficient in English. Project coordination was led by Prof. Renata Vieira and Prof. Rafael Prikladnicki (PUCRS).
                \item \textit{Role:} Scientific coordinator for the University of Bari unit. Designed and implemented an open-source real-time translation system for synchronous communication in distributed software requirements meetings. Developed comprehensive experimental methodology to evaluate machine translation effectiveness compared to native language usage during requirements elicitation and negotiation meetings.
            \end{highlights}
        \end{onecolentry}


        \vspace{0.2 cm}

        \begin{twocolentry}{
            Nov 2011 – Nov 2013
        }
            \textbf{INTERSOCIAL - Unleashing the Power of Social Networking for Enhancing Regional Systems}\end{twocolentry}

        \vspace{0.10 cm}
        \begin{onecolentry}
            \begin{highlights}
                \item \textit{Funded by:} EU INTERREG Greece-Italy 2007-13, Priority Axis 1: Strengthening competitiveness and innovation
                \item \textit{Partners:} University of Ioannina (project coordination), University of Bari, University of Patras, Euro-Mediterranean Cultural Heritage Agency
                \item \textit{Description:} European project that aimed at fostering regional development through the creation of a cross-border network of social innovation and entrepreneurship. The project developed a social networking platform to facilitate knowledge sharing and collaboration among regional stakeholders, enhancing SME competitiveness through social networking technologies and business data analytics. Project coordination was led by Prof. Evaggelia Pitoura.
                \item \textit{Role:} Work package leader. Led operational teams across two work packages: conducted a state-of-the-art analysis of social networking tools and enterprise-level social web presence policies (WP3: Development of Innovation Devices), and managed requirements analysis, deployment, and experimental evaluation of ESA (Enterprise Social Aggregator), a custom social networking tool for SMEs (WP4: Deployment and evaluation of innovation devices). The University of Bari unit was coordinated by Prof. Filippo Lanubile.
            \end{highlights}
        \end{onecolentry}


        \vspace{0.2 cm}

        \begin{twocolentry}{
            June 2012 – Oct 2015
        }
            \textbf{LOGIN - LOgistica INtegrata}\end{twocolentry}

        \vspace{0.10 cm}
        \begin{onecolentry}
            \begin{highlights}
                \item \textit{Funded by:} MISE - call Industria 2015 - Made in Italy New Technologies Program 2012-2015 (MI01\_00294)
                \item \textit{Partners:} DAISY-NET s.c.a.r.l (project coordination), University of Bari, Politecnico di Bari, University of Salento, University of Foggia, Cetma
                \item \textit{Description:} National project that developed an integrated logistics platform based on Service Oriented Architecture (SOA)for tracking goods movement and monitoring related information flows through web services.
                \item \textit{Role:} Work package leader. Led work packages focused on the CollabWeb component: defined specifications within the integrated LOGIN model (WP2) and designed component architecture (WP3). Project coordination for the University of Bari unit was led by Prof. Giuseppe Visaggio.
            \end{highlights}
        \end{onecolentry}



    
    \section{Funding Acquisition Proposals}



        
        \begin{onecolentry}
            \textbf{DisTrac: A tool for tracking disengagement in open-source software projects}\end{onecolentry}

        \vspace{0.10 cm}
        \begin{onecolentry}
            \begin{highlights}
                \item \textit{In response:} International call \href{https://new.nsf.gov/funding/opportunities/dcl-nsf-italian-ministry-universities-research-lead-agency}{`NSF-MUR Lead Agency Opportunity in Artificial Intelligence'}
                \item \textit{Requested budget:} € 215,380 (MUR funding) + \$600,000 (NSF funding)
                \item \textit{Partners:} North Arizona University (Lead), Colorado State University, University of Bari
                \item \textit{Description:} Development of tools and methodologies to track and analyze developer disengagement patterns in open-source projects
                \item \textit{Role:} PI for Italian unit
                \item \textit{Duration:} 36 months
                \item \textit{Status:} Under review at NSF
            \end{highlights}
        \end{onecolentry}


        \vspace{0.2 cm}

        \begin{onecolentry}
            \textbf{ARIANNA: ARtIficiAl iNtelligeNce for virtuAl meetings}\end{onecolentry}

        \vspace{0.10 cm}
        \begin{onecolentry}
            \begin{highlights}
                \item \textit{In response to:} Regional call \href{https://www.regione.puglia.it/web/ricerca-e-relazioni-internazionali/-/reti}{`Reti - Support for collaboration between companies and research organizations'}
                \item \textit{Requested budget:} € 1,000,000 total (€ 250,000 for PeoplewareAI unit)
                \item \textit{Partners:} Quavlive s.r.l. (Lead), PeoplewareAI s.r.l., Politecnico di Bari
                \item \textit{Description:} Development of an AI-powered videoconferencing platform with marketplace for intelligent applications, focusing on virtual assistants and e-learning capabilities
                \item \textit{Role:} Co-PI, leading the development of AI applications for e-learning
                \item \textit{Duration:} 24 months (Jan 2025 - Dec 2026)
                \item \textit{Status:} Under review
            \end{highlights}
        \end{onecolentry}



    
    \section{Academic Service}



        
        \begin{onecolentry}
            \textbf{Rector’s Delegate for the GARR Network}\end{onecolentry}

        \vspace{0.10 cm}
        \begin{onecolentry}
            \begin{highlights}
                \item \textit{May 2023 - present}: The GARR Consortium is the Italian national research and education network, providing high-speed Internet connectivity and advanced services to universities and research institutions.
            \end{highlights}
        \end{onecolentry}


        \vspace{0.2 cm}

        \begin{onecolentry}
            \textbf{CS Dept. Director’s Delegate for Internship Programs}\end{onecolentry}

        \vspace{0.10 cm}
        \begin{onecolentry}
            \begin{highlights}
                \item \textit{Mar 2023 - present}: Responsibilities include overseeing the development and implementation of internship programs, coordinating with industry partners, mentoring students, ensuring compliance with academic standards, and evaluating the effectiveness of the internship experiences.
            \end{highlights}
        \end{onecolentry}


        \vspace{0.2 cm}

        \begin{onecolentry}
            \textbf{Associate Editor}\end{onecolentry}

        \vspace{0.10 cm}
        \begin{onecolentry}
            \begin{highlights}
                \item Elsevier \href{https://www.sciencedirect.com/journal/journal-of-systems-and-software/about/editorial-board}{\textit{Journal of Systems and Software (JSS)}}, rank: SJR Q1
                \item Springer \textit{\href{https://link.springer.com/journal/10515/editorial-board}{Automated Software Engineering (ASE)}}
            \end{highlights}
        \end{onecolentry}


        \vspace{0.2 cm}

        \begin{onecolentry}
            \textbf{Guest Editor}\end{onecolentry}

        \vspace{0.10 cm}
        \begin{onecolentry}
            \begin{highlights}
                \item \textit{2024}: Springer \textit{Empirical Software Engineering (EMSE)}, rank: SJR Q1. Special Issue on ``Software Maintenance and Evolution.'' Co-editor: Sarah Nadi (New York University, Abu Dhabi)
                \item \textit{2023}: Springer \textit{Empirical Software Engineering’ (EMSE}), rank: SJR Q1. Special Issue on \href{https://emsejournal.github.io/special_issues/2023_SI_CHASE.html}{``Cooperative and Human Aspects of Software Engineering''}. Co-editors: Hourieh Khalajzadeh (Deakin University, Australia) and Igor Steinmacher (NAU, USA)
                \item \textit{2021}: Elsevier \textit{Journal of System and Software (JSS)}, rank: SJR Q1. Special Issue on \href{https://www.sciencedirect.com/journal/journal-of-systems-and-software/special-issue/10V2WHQCGWT}{``Global Software Engineering: Challenges and Solutions.''} Vol. 174, Apr. 2021. Co-editors: Alpana Dubey (Accenture Labs, India), Christof Ebert (Vector Consulting, Germany), Paolo Tell (IT University of Copenhagen, Denmark)
            \end{highlights}
        \end{onecolentry}


        \vspace{0.2 cm}

        \begin{onecolentry}
            \textbf{Review Board Member}\end{onecolentry}

        \vspace{0.10 cm}
        \begin{onecolentry}
            \begin{highlights}
                \item \textit{2023 – present}: Replicated Computational Results Distinguished Reviewers Board for ACM \textit{Transactions on Software Engineering and Methodology (TOSEM)}, rank: SJR Q1
                \item \textit{2020 – 2022}: ACM \textit{Transactions on Software Engineering and Methodology (TOSEM)}, rank: SJR Q1
                \item \textit{2015 – 2019}: Springer \textit{Journal of Empirical Software Engineering (EMSE)}, rank: SJR Q1
            \end{highlights}
        \end{onecolentry}


        \vspace{0.2 cm}

        \begin{onecolentry}
            \textbf{Peer Reviews (partial list)} -- \textbf{verified record on \href{https://www.webofscience.com/wos/author/record/H-4177-2014}{Clarivate}}\end{onecolentry}

        \vspace{0.10 cm}
        \begin{onecolentry}
            \begin{highlights}
                \item (23) Springer \textit{Journal of Empirical Software Engineering (EMSE)}, rank: SJR Q1
                \item (19) IEEE \textit{Transactions on Software Engineering (TSE)}, rank: SJR Q1
                \item (16) ACM \textit{Transactions on Software Engineering and Methodology (TOSEM)}, rank: SJR Q1
                \item (12) Elsevier \textit{Journal of System and Software (JSS)}, rank: SJR Q1
                \item (8) Elsevier \textit{Information and Software Technolog (INFSOF)}, rank: SJR Q1
                \item (7) Wiley \textit{Journal of Software: Evolution and Process (JSEP)}, rank: SJR Q2
                \item (5) IEEE \textit{Software}, rank: SJR Q2
                \item (2) IEEE \textit{Transactions on Affective Computing}, rank: SJR Q1
                \item (1) ACM \textit{Transactions on Internet Technologies (TOIT)}, rank: SJR Q1
            \end{highlights}
        \end{onecolentry}



    
    \section{Events Organization}



        
        \begin{onecolentry}
            \textbf{Program Co-Chair}\end{onecolentry}

        \vspace{0.10 cm}
        \begin{onecolentry}
            \begin{highlights}
                \item \textit{40th Int’l Conf. on Software Maintenance and Evolution (ICSME’24)}, Flagstaff, AZ, USA, Oct 2024, \href{https://portal.core.edu.au/conf-ranks/676/}{iCORE rank: A}
                \item \textit{16th Int’l Conf. on Cooperative and Human Aspects of Software Engineering (CHASE’23)}, Melbourne, Australia, May 2023
                \item \textit{8th Int’l Workshop on Social Software Engineering (SSE’16)}, Seattle, WA, USA, Nov 14, 2016 – co-located with \textit{FSE'16}
                \item \textit{1st Int’l Workshop on Trust in Virtual Teams: Theory and Tools}, San Antonio, TX, USA, Feb 24, 2013 – co-located with \textit{CSCW'13}
            \end{highlights}
        \end{onecolentry}


        \vspace{0.2 cm}

        \begin{onecolentry}
            \textbf{General Chair}\end{onecolentry}

        \vspace{0.10 cm}
        \begin{onecolentry}
            \begin{highlights}
                \item \textit{14th Int'l Conf. on Global Software Engineering (ICGSE'19)}, Montreal, Canada, 25-26 May 2018, \href{https://portal.core.edu.au/conf-ranks/650/}{ICORE rank: C}
            \end{highlights}
        \end{onecolentry}


        \vspace{0.2 cm}

        \begin{onecolentry}
            \textbf{Steering Board Member}\end{onecolentry}

        \vspace{0.10 cm}
        \begin{onecolentry}
            \begin{highlights}
                \item \textit{Int'l Conf. on Global Software Engineering (ICGSE)}, 2019-2022, \href{https://portal.core.edu.au/conf-ranks/650/}{ICORE rank: C}
            \end{highlights}
        \end{onecolentry}


        \vspace{0.2 cm}

        \begin{onecolentry}
            \textbf{Track Co-Chair}\end{onecolentry}

        \vspace{0.10 cm}
        \begin{onecolentry}
            \begin{highlights}
                \item \textit{43rd Int’l Conf. on Software Engineering (ICSE’21), Student Contest on Software Engineering (SCORE) track}, Madrid, Spain, May 2021, \href{https://portal.core.edu.au/conf-ranks/1209/}{ICORE rank: A*}
            \end{highlights}
        \end{onecolentry}


        \vspace{0.2 cm}

        \begin{onecolentry}
            \textbf{Workshops \& Tutorials Co-Chair}\end{onecolentry}

        \vspace{0.10 cm}
        \begin{onecolentry}
            \begin{highlights}
                \item \textit{23rd Int’l Conf. on Product-Focused Software Process Improvement (PROFES’23)}, Jyväskylä, Finland, 21-23 November 2022, \href{https://portal.core.edu.au/conf-ranks/1696/}{ICORE rank: B}
            \end{highlights}
        \end{onecolentry}


        \vspace{0.2 cm}

        \begin{onecolentry}
            \textbf{Workshops Co-Chair}\end{onecolentry}

        \vspace{0.10 cm}
        \begin{onecolentry}
            \begin{highlights}
                \item \textit{10th Int’l Conf. on Global Software Engineering (ICGSE'15)}, Ciudad Real, Spain, 13-16 July 2015, \href{https://portal.core.edu.au/conf-ranks/650/}{ICORE rank: C}
            \end{highlights}
        \end{onecolentry}


        \vspace{0.2 cm}

        \begin{onecolentry}
            \textbf{Open-science Co-Chair}\end{onecolentry}

        \vspace{0.10 cm}
        \begin{onecolentry}
            \begin{highlights}
                \item \textit{15th Int'l Symposium on Empirical Software Engineering and Measurement (ESEM’21)}, Bari, Italy, 11-15 Oct 2021, \href{https://portal.core.edu.au/conf-ranks/1376/}{ICORE rank: A}
                \item \textit{14th Int'l Symposium on Empirical Software Engineering and Measurement (ESEM’20)}, Bari, Italy, 5-9 Oct 2020, \href{https://portal.core.edu.au/conf-ranks/1376/}{ICORE rank: A}
            \end{highlights}
        \end{onecolentry}


        \vspace{0.2 cm}

        \begin{onecolentry}
            \textbf{Publicity \& Social Media Chair}\end{onecolentry}

        \vspace{0.10 cm}
        \begin{onecolentry}
            \begin{highlights}
                \item \textit{21st Int’l Conf. on Agile Software Development (XP’20)}, Copenhagen, Denmark, 8-12 June 2020, \href{https://portal.core.edu.au/conf-ranks/355/}{ICORE rank: B}
                \item \textit{13th Int’l Conf. on Global Software Engineering (ICGSE'18)}, Gothenburg, Sweden, 28-29 May 2018, \href{https://portal.core.edu.au/conf-ranks/650/}{ICORE rank: C}
                \item \textit{8th Int’l Conf. on Global Software Engineering (ICGSE'13)}, Bari, Italy, 26-29 Aug 2013, \href{https://portal.core.edu.au/conf-ranks/650/}{ICORE rank: C}
            \end{highlights}
        \end{onecolentry}



    
    \section{Keynote Presentations}



        
        \begin{twocolentry}{
            Sept 2021
        }
            \textbf{The Potential and Challenges of Personality Detection in Software Engineering Research} -- \textit{4th Int'l Workshop on Affective Computing in Requirements Engineering (AffectRE’21)}, co-located with \textit{RE’21}\end{twocolentry}



        \vspace{0.2 cm}

        \begin{twocolentry}{
            Aug 2016
        }
            \textbf{Facing Communication Challenges in Distributed Software Development} -- \textit{1st Int’l Workshop on Virtual Teams: Experiences in Global Software Engineering (VirtuES ’13)}, co-located with \textit{ICGSE’13}\end{twocolentry}




    
    \section{Membership in Program Committees}

    \begin{onecolentry}
        \begin{highlightsforbulletentries}


        \item \textit{32nd IEEE Int’l Conf. on Software Analysis, Evolution, and Reengineering (SANER’25)}, Montréal, Canada 4-7 Mar. 2025, \href{https://portal.core.edu.au/conf-ranks/2280/}{ICORE rank: A}

        \item \textit{47th IEEE/ACM Int'l Conf. on Software Engineering (ICSE'25) – SEIS track}, Ottawa, Canada, April 27-May 3 2025, \href{https://portal.core.edu.au/conf-ranks/1209/}{ICORE rank: A*}

        \item \textit{47th IEEE/ACM Int'l Conf. on Software Engineering (ICSE'25) – Workshops track,} Ottawa, Canada, April 27-May 3 2025, \href{https://portal.core.edu.au/conf-ranks/1209/}{ICORE rank: A*}

        \item \textit{46th IEEE/ACM Int'l Conf. on Software Engineering (ICSE'24) – SEIS track}, Lisbon, Portugal, 14-20 Apr. 2024, \href{https://portal.core.edu.au/conf-ranks/1209/}{ICORE rank: A*}

        \item \textit{46th IEEE/ACM Int'l Conf. on Software Engineering (ICSE'24) – Workshops track}, Lisbon, Portugal, 14-20 Apr. 2024, \href{https://portal.core.edu.au/conf-ranks/1209/}{ICORE rank: A*}

        \item \textit{46th IEEE/ACM Int'l Conf. on Software Engineering (ICSE'24) – Student Research Competition (SRC) track}, Lisbon, Portugal, 14-20 Apr. 2024, \href{https://portal.core.edu.au/conf-ranks/1209/}{ICORE rank: A*}

        \item \textit{18th ACM/IEEE Int'l Symposium on Empirical Software Engineering and Measurement (ESEM'24)}, Barcelona, Spain 24-25 Oct. 2024, \href{https://portal.core.edu.au/conf-ranks/1376/}{ICORE rank: A}

        \item \textit{21st Int'l Conf. on Mining Software Repositories (MSR'24) – Mining Challenge track}, Lisbon, Portugal, 15-16 Apr. 2024, \href{https://portal.core.edu.au/conf-ranks/711/}{ICORE rank: A}

        \item \textit{3rd Int'l Conf. on AI Engineering – Software Engineering for AI (CAIN'24)}, Lisbon, Portugal, 14-15 Apr. 2024

        \item \textit{1st IEEE/ACM Workshop on Multi-disciplinary, Open, and RElevant Requirements Engineering (MO2RE'24)}, Lisbon Portugal 16 Apr. 2024

        \item \textit{17th ACM/IEEE Int'l Symposium on Empirical Software Engineering and Measurement (ESEM'23)}, New Orleans, USA 26-27 Oct. 2023, \href{https://portal.core.edu.au/conf-ranks/1376/}{ICORE rank: A}

        \item \textit{39th Int'l Conf. on Software Maintenance and Evolution (ICSME'23) – NIER track}, Bogotà, Colombia, 1-6 Oct. 2023, \href{https://portal.core.edu.au/conf-ranks/676/}{ICORE rank: A}

        \item \textit{45th IEEE/ACM Int'l Conf. on Software Engineering – Posters track (ICSE'23)}, Melbourne, Australia, 14-20 May 2023, \href{https://portal.core.edu.au/conf-ranks/1209/}{ICORE rank: A*}

        \item \textit{23rd Int'l Conf. on Product-Focused Software Process Improvement (PROFES'23)}, Dornbirn, Austria 10-13 Dec. 2023, \href{https://portal.core.edu.au/conf-ranks/1696/}{ICORE rank: B}

        \item \textit{30th IEEE/ACM Int'l Conf. on Program Comprehension – ERA track (ICPC'22)}, Pittsburgh, USA, 16–17 May 2022, \href{https://portal.core.edu.au/conf-ranks/1181/}{ICORE rank: A}

        \item \textit{17th Int'l Conf. on Global Software Engineering (ICSSP/ICGSE'22)}, Pittsburgh, USA, 19–20 May 2022, \href{https://portal.core.edu.au/conf-ranks/650/}{ICORE rank: C}

        \item \textit{15th IEEE/ACM Int'l Conf. on Cooperative and Human Aspects of Software Engineering (CHASE'22)} Pittsburgh, USA, 18–19 May 2022

        \item \textit{1st IEEE/ACM Workshop on Natural Language-based Software Engineering (NLBSE'22)}, Pittsburgh, USA, 8 May 2022

        \item \textit{29th IEEE Int'l Conf. on Software Analysis, Evolution and Reengineering (SANER'22)}, Honolulu, USA, 15 Mar. 2022, \href{https://portal.core.edu.au/conf-ranks/2280/}{ICORE rank: A}

        \item \textit{23rd Int'l Conf. on Agile Software Development (XP'22)}, Copenhagen, Denmark, 13-17 June 2022, \href{https://portal.core.edu.au/conf-ranks/355/}{ICORE rank: B}

        \item \textit{16th ACM/IEEE Symposium on Empirical Software Engineering and Measurement (ESEM'21) - Emerging Results and Vision track}, online, 14-15 Oct. 2021, \href{https://portal.core.edu.au/conf-ranks/1376/}{ICORE rank: A}

        \item \textit{22nd Int'l Conf. on Agile Software Development (XP'21)}, online, 14-18 June 2021, \href{https://portal.core.edu.au/conf-ranks/355/}{ ICORE rank: B}

        \item \textit{37th IEEE Int'l Conf. on Software Maintenance and Evolution (ICSME'21) - NIER Track}, online, 27 Sept. - 1 Oct. 2021, \href{https://portal.core.edu.au/conf-ranks/676/}{ICORE rank: A}

        \item \textit{14th ACM/IEEE Int'l Conf. on Cooperative and Human Aspects of Software Engineering (CHASE'21)}, online, 20-21 May 2021

        \item 17th Int'l Conf. on Open-Source Software (\textit{OSS'21}), online, 12 May 2021

        \item \textit{16th ACM/IEEE Int'l Conf. on Global Software Engineering (ICSSP/ICGSE'21)}, online, 18-19 May 2021,' \href{https://portal.core.edu.au/conf-ranks/650/}{ICORE rank: C}

        \item \textit{17th IEEE/ACM Int'l Conf. on Mining Software Repositories (MSR'20) – Mining Challenge track}, online, 29-30 June 2020, \href{https://portal.core.edu.au/conf-ranks/711/}{ICORE rank: A}

        \item \textit{15th ACM/IEEE Symposium on Empirical Software Engineering and Measurement (ESEM'20) - Emerging Results and Vision track}, online, 5-7 Oct. 2020, \href{https://portal.core.edu.au/conf-ranks/1376/}{ICORE rank: A}

        \item \textit{ACM Joint European Software Engineering Conference and Symposium on the Foundations of Software Engineering (ESEC/FSE'20) - SRC track}, online, 6-16 Nov. 2020, \href{https://portal.core.edu.au/conf-ranks/52/}{ICORE rank: A*}

        \item \textit{21st Int'l Conf. on Agile Software Development (XP'20)}, online, 8-12 June 2020, \href{https://portal.core.edu.au/conf-ranks/355/}{ICORE rank: B}

        \item \textit{15th IEEE/ACM Int'l Conf. on Global Software Engineering (ICGSE'20)}, online, 26-28 June 2020, \href{https://portal.core.edu.au/conf-ranks/650/}{ICORE rank: C}

        \item \textit{13th IEEE/ACM Int'l Conf. on Cooperative and Human Aspects of Software Engineering (CHASE'20)}, online, 1-2 July 2020

        \item \textit{36th Int'l Conf. on Software Maintenance and Evolution (ICSME'20) Doctoral Symposium Program Committee}, Adelaide, Australia, Sep 28-Oct 02, 2020, \href{https://portal.core.edu.au/conf-ranks/676/}{ICORE rank: A}

        \item \textit{13th Int'l Conf. on the Quality of Information and Communications Technology (QUATIC'20)}, online, 9-11 Sept. 2020, \href{https://portal.core.edu.au/conf-ranks/2180/}{ICORE rank: C}

        \item \textit{16th Int'l Symposium on Open Collaboration (OpenSym'20)}, online, 26–27 Aug. 2020, \href{https://portal.core.edu.au/conf-ranks/1432/}{ICORE rank: C}

        \item \textit{15th Int'l Conf. on Global Software Engineering (ICGSE'20)}, online, 26-28 June 2020, \href{https://portal.core.edu.au/conf-ranks/650/}{ICORE rank: C}

        \item \textit{27th IEEE Int'l Conf. on Software Analysis, Evolution and Reengineering (SANER'20)}, London, Ontario, Canada, February 18-21, 2020, \href{https://portal.core.edu.au/conf-ranks/2280/}{ICORE rank: A}

        \item \textit{13th Int'l Symposium on Empirical Software Engineering and Measurement (ESEM'19)}, Porto de Galinhas, Brazil, 16-20 Sept. 2019, \href{https://portal.core.edu.au/conf-ranks/1376/}{ICORE rank: A}

        \item \textit{20th Int'l Conf. on Agile Software Development (XP'19)}, Montréal, Canada, 21-25 May 2019, \href{https://portal.core.edu.au/conf-ranks/355/}{ICORE rank: B}

        \item \textit{12th Int'l Symposium on Empirical Software Engineering and Measurement (ESEM'18)}, Oulu, Finland, 11-12 Oct. 2018, \href{https://portal.core.edu.au/conf-ranks/1376/}{ICORE rank: A}

        \item \textit{12th Workshop on Distributed Software Development, Software Ecosystems and Systems-of-Systems (WDES'18)}, Madrid, Spain, Sept. 24, 2018

        \item \textit{40th Int'l Conf. on Software Engineering (ICSE'18) – Student Contest on Software Engineering (SCORE) track}, Gothenburg, Sweden, May 27–June 3, 2018, \href{https://portal.core.edu.au/conf-ranks/1209/}{ICORE rank: A*}

        \item \textit{13th Int'l Conf. on Global Software Engineering (ICGSE'18)}, Gothenburg, Sweden, 24-26 May 2018, \href{https://portal.core.edu.au/conf-ranks/650/}{ICORE rank: C}

        \item \textit{1st Int'l Workshop on Affective Computing for Requirements Engineering (AffectRE'18)}, Banff, Canada Aug 21, 2018

        \item \textit{11th Int'l Symposium on Empirical Software Engineering and Measurement (ESEM'17)} Toronto, Canada, 9-10 Nov. 2017, \href{https://portal.core.edu.au/conf-ranks/1376/}{ICORE rank: A}

        \item \textit{12th Int'l Conf. on Global Software Engineering (ICGSE'17)}, Buenos Aires, Argentina, 22-23 May 2017, \href{https://portal.core.edu.au/conf-ranks/650/}{ICORE rank: C}

        \item \textit{10th Int'l Symposium on Empirical Software Engineering and Measurement (ESEM'16)}, Ciudad Real, Spain, 8-9 Sept. 2016, \href{https://portal.core.edu.au/conf-ranks/1376/}{ICORE rank: A}

        \item \textit{11th Int'l Conf. on Global Software Engineering (ICGSE'16)}, Orange County, CA, USA, 2-5 Aug. 2016, \href{https://portal.core.edu.au/conf-ranks/650/}{ICORE rank: C}

        \item \textit{9th Int'l Symposium on Empirical Software Engineering and Measurement (ESEM'15)}, Beijing, China, 22-23 Oct. 2015, \href{https://portal.core.edu.au/conf-ranks/1376/}{ICORE rank: A}

        \item \textit{10th Int'l Conf. on Global Software Engineering (ICGSE'15)}, Ciudad Real, Spain, 13-16 July 2015, \href{https://portal.core.edu.au/conf-ranks/650/}{ICORE rank: C}

        \item \textit{7th Software Quality Days (SWQD'15)}, Vienna, Austria, 20-23 Jan. 2015

        \item \textit{6th Software Quality Days (SWQD'14)}, Vienna, Austria, 14-16 Jan. 2014

        \item \textit{8th Int'l Conf. on Global Software Engineering (ICGSE'13)}, Bari, Italy, 26-29 Aug. 2013, \href{https://portal.core.edu.au/conf-ranks/650/}{ICORE rank: C}

        \item \textit{7th Int'l Conf. on Global Software Engineering (ICGSE'12)}, Porto Alegre, Brazil, 27-30 Aug. 2012, \href{https://portal.core.edu.au/conf-ranks/650/}{ICORE rank: C}

        \item \textit{5th Workshop of the Italian Eclipse Community (Eclipse-IT'10)}, Savona, Italy, Sep. 30-Oct. 1, 2010

        \item \textit{5th Int'l Conf. on P2P, Parallel, Grid, Cloud and Internet Computing (3PGCIC-2010)}, Fukuoka, Japan, 4-6 Nov. 2010

        \item \textit{2nd Int'l Workshop on Social Software Engineering and Applications (SoSEA'09)}, Amsterdam, The Netherlands, 24 Aug. 2009

        \item \textit{4th Int'l Conf. on Global Software Engineering (ICGSE'09)}, Limerick, Ireland, 13-16 Jul. 2009, \href{https://portal.core.edu.au/conf-ranks/650/}{ICORE rank: C}

        \item \textit{4th Workshop of the Italian Eclipse Community (Eclipse-IT'09)}, Bergamo, Italy, 28-29 Sept. 2009

        \item \textit{14th Collaboration Researchers' International Workshop on Groupware (CRIWG'08)}, Omaha, NE, USA, 14-18 Sept. 2008

        \item \textit{3rd Int'l Conf. on Global Software Engineering (ICGSE'08)}, Bangalore, India, 17-20 Aug. 2008, \href{https://portal.core.edu.au/conf-ranks/650/}{ICORE rank: C}

        \item \textit{3rd Workshop of the Italian Eclipse Community (Eclipse-IT'08)}, Bari, Italy, 17-18 Nov. 2008

        \item \textit{1st Int'l Conf. on Software Engineering and Applications (ICSEA'06)}, Tahiti, Polynesia, Oct. 29 – Nov. 3, 2006, \href{https://portal.core.edu.au/conf-ranks/1210/}{ICORE rank: C}


        \end{highlightsforbulletentries}
    \end{onecolentry}

    \section{Membership in Doctoral Boards}



        
        \begin{twocolentry}{
            2024 – 2025
        }
            \textbf{Ph.D. program in Computer Science and Mathematics (XL cycle)} -- University of Bari, Italy\end{twocolentry}



        \vspace{0.2 cm}

        \begin{twocolentry}{
            2023 – 2024
        }
            \textbf{Ph.D. program in Computer Science and Mathematics (XXXIX cycle)} -- University of Bari, Italy\end{twocolentry}



        \vspace{0.2 cm}

        \begin{twocolentry}{
            2022 – 2023
        }
            \textbf{Ph.D. program in Computer Science and Mathematics (XXXVIII cycle)} -- University of Bari, Italy\end{twocolentry}



        \vspace{0.2 cm}

        \begin{twocolentry}{
            2021 – 2022
        }
            \textbf{Interuniversity Ph.D. program in Aerospace Engineering and Sciences (XXXVII cycle)} -- Polytechnic University of Bari, Italy\end{twocolentry}



        \vspace{0.2 cm}

        \begin{twocolentry}{
            2020 – 2021
        }
            \textbf{Interuniversity Ph.D. program in Aerospace Engineering and Sciences (XXXVI cycle)} -- Polytechnic University of Bari, Italy\end{twocolentry}




    
    \section{Research Visits}



        
        \begin{twocolentry}{
            Oct 2024 – Oct 2024
        }
            \textbf{M-Group, Northern Arizona Univ. (NAU)} -- Flagstaff, AZ, USA\end{twocolentry}

        \vspace{0.10 cm}
        \begin{onecolentry}
            \begin{highlights}
                \item \textit{Hosts}: Prof. Marco Aurelio Gerosa, Prof. Igor Steinmacher
            \end{highlights}
        \end{onecolentry}


        \vspace{0.2 cm}

        \begin{twocolentry}{
            Sept 2023 – Sept 2023
        }
            \textbf{M-Group, University of Oulu} -- Oulu, Finland\end{twocolentry}

        \vspace{0.10 cm}
        \begin{onecolentry}
            \begin{highlights}
                \item \textit{Host}: Prof. Davide Taibi
            \end{highlights}
        \end{onecolentry}


        \vspace{0.2 cm}

        \begin{twocolentry}{
            July 2017 – Oct 2017
        }
            \textbf{STRUDEL, Carnegie Mellon University (CMU)} -- Pittsburgh, USA\end{twocolentry}

        \vspace{0.10 cm}
        \begin{onecolentry}
            \begin{highlights}
                \item \textit{Host}: Prof. Bogdan Vasilescu
            \end{highlights}
        \end{onecolentry}


        \vspace{0.2 cm}

        \begin{twocolentry}{
            Nov 2016 – Dec 2016
        }
            \textbf{CHISEL, University of Victoria} -- Victoria, Canada\end{twocolentry}

        \vspace{0.10 cm}
        \begin{onecolentry}
            \begin{highlights}
                \item \textit{Host}: Prof. Margaret-Anne Storey
            \end{highlights}
        \end{onecolentry}


        \vspace{0.2 cm}

        \begin{twocolentry}{
            Jan 2006 – Apr 2006
        }
            \textbf{SEGAL, University of Victoria} -- Victoria, Canada\end{twocolentry}

        \vspace{0.10 cm}
        \begin{onecolentry}
            \begin{highlights}
                \item \textit{Host}: Prof. Daniela Damian
            \end{highlights}
        \end{onecolentry}


        \vspace{0.2 cm}

        \begin{twocolentry}{
            May 2005 – June 2005
        }
            \textbf{Distributed Systems Group, Vienna University of Technology} -- Vienna, Austria\end{twocolentry}

        \vspace{0.10 cm}
        \begin{onecolentry}
            \begin{highlights}
                \item \textit{Host}: Prof. Schaharm Dustdar
            \end{highlights}
        \end{onecolentry}



    
    \section{Invited Seminars}



        
        \begin{twocolentry}{
            Oct 2017
        }
            \textbf{Collaboration in Software Engineering: 10 Years in Review} -- Carnegie Mellon University, USA\end{twocolentry}



        \vspace{0.2 cm}

        \begin{twocolentry}{
            Nov 2016
        }
            \textbf{Classification Models in Software Engineering: From Defect to Best-Answer Prediction} -- University of Victoria, Canada\end{twocolentry}




    
    \section{Software}



        
        \begin{onecolentry}
            \textbf{BehaViz™}\end{onecolentry}

        \vspace{0.10 cm}
        \begin{onecolentry}
            \begin{highlights}
                \item The flagship product of PeoplewareAI focused on behavioral data analysis and emotion recognition from written text
                \item Features customizable emotion recognition through both custom annotated datasets and classification models for improved accuracy and flexibility
                \item Deployed as a cloud-based SaaS solution with secure integrations to customer data sources, ensuring data privacy by processing without data movement
            \end{highlights}
        \end{onecolentry}


        \vspace{0.2 cm}

        \begin{onecolentry}
            \textbf{MLOps Pipeline}\end{onecolentry}

        \vspace{0.10 cm}
        \begin{onecolentry}
            \begin{highlights}
                \item End-to-end ML workflow solution handling project documentation, version control, experiment tracking, and data quality assurance
                \item Built-in CI/CD capabilities for seamless ML model deployment and iteration
                \item REST API integration for ML components with containerization and orchestration
                \item Production monitoring capabilities for ML components performance tracking
                \item Comprehensive solution for ML model development and deployment at scale
                \item Domain-specific MLOps solutions for healthcare, implementing regulation-compliant data pipelines and ensuring patient data privacy
            \end{highlights}
        \end{onecolentry}


        \vspace{0.2 cm}

        \begin{twocolentry}{
            \href{https://github.com/collab-uniba/EMTk}{EMTk on GitHub}
        }
            \textbf{EMTk (Emotion-Mining Toolkit)}\end{twocolentry}

        \vspace{0.10 cm}
        \begin{onecolentry}
            \begin{highlights}
                \item EmoTxt is the module for training custom emotion classifiers from text. It provides out-of-the-box an emotion classifier specifically tuned for mining emotion from developers' communication channels, such as Stack Overflow.
                \item Senti4SD is the module for emotion-polarity classification (sentiment analysis) specifically trained on technical corpora from developers' communication channels, such as GitHub and Stack Oveflow.
            \end{highlights}
        \end{onecolentry}



    
    \section{References}

    \begin{onecolentry}
        \begin{highlightsforbulletentries}


        \item Prof. Marco Aurelio Gerosa (Northern Arizona University) - \href{mailto:marco.gerosa@nau.edu}{marco.gerosa@nau.edu}

        \item Prof. Igor Steinmacher (Northern Arizona University) - \href{mailto:igor.steinmacher@nau.edu}{igor.steinmacher@nau.edu}

        \item Prof. Bogdan Vasilescu (Carnegie Mellon University) - \href{mailto:bogdanv@andrew.cmu.edu}{bogdanv@andrew.cmu.edu}

        \item Prof. Marcos Kalinowski (PUC-Rio) - \href{mailto:kalinowski@inf.puc-rio.br}{kalinowski@inf.puc-rio.br}

        \item Prof. Christoph Treude (Singapore Management University) - \href{mailto:ctreude@smu.edu.sg}{ctreude@smu.edu.sg}

        \item Prof. Darja Šmite (BTH) - \href{mailto:darja.smite@bth.se}{darja.smite@bth.se}


        \end{highlightsforbulletentries}
    \end{onecolentry}


\end{document}